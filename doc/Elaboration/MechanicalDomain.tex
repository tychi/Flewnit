

\label{sec:mechanicalDomain}
blindtext blindtext blindtext blindtext blindtext blindtext blindtext blindtext blindtext blindtext blindtext blindtext blindtext blindtext blindtext 
	
\subsubsection{Fluidsimulation}
blindtext blindtext blindtext blindtext blindtext blindtext blindtext blindtext blindtext blindtext blindtext 		
blindtext blindtext blindtext blindtext blindtext blindtext blindtext blindtext blindtext blindtext blindtext 
blindtext blindtext blindtext blindtext blindtext 
		
	\sectionlevelfour{Grundlagen}
	blindtext blindtext blindtext blindtext blindtext blindtext blindtext blindtext 
	
		\sectionlevelfive{Die Navier-Stokes-Gleichungen}
		Herleitung, Erläuterung
		blindtext blindtext blindtext blindtext blindtext blindtext blindtext blindtext blindtext blindtext 
			
		\sectionlevelfive{Grid-basierte vs. Partikelbasierte Simulation}
		\label{par:GridVsParticle}
		blindtext blindtext blindtext blindtext blindtext blindtext blindtext blindtext blindtext blindtext blindtext 
		blindtext blindtext blindtext 
			
			\sectionlevelsix{Die zwei Sichtweisen: Lagrange vs. Euler}
			\label{par:lagrangeEuler}
			blindtext blindtext blindtext blindtext blindtext blindtext 
				
		\sectionlevelfive{Smoothed Particle Hydrodynamics}
		ursprünglich aus astronomie blubb blubb 
		\todo{überlegen, ob ich aus Interesse nicht noch weiter in die Richtung recherchieren sollte, da ich nach meiner 
		Implementierung erst so richtig beeindruckt von dem Verfahren war (ich habe im Internet noch keine Fluid-Demo 
		gefunden, die ebenfalls SPH implementiert; ok., ich hab auch nicht gesucht ;) ), und gerne mehr über die 
		Hintergründe verstehen würde... problem, wie immer: Zeitdruck ;( }		
		
	\sectionlevelfour{Verwandte Arbeiten}
	\label{sec:relatedWork}
	Referenzen auf Müller03, Thomas Steil, Goswami, GPU gems, Aufzeigen, was ich von wem übernommen habe, was ich selbst  
	modifiziert habe aufgrund von etwaigen Fehlern in den Papers odel weil OpenCL es schlciht nicht zulässt;
	
	\sectionlevelfour{Umfang}
		Abgrenzungf zwischen bisher funktionierenden Features, bisher programmierten, aber nicht integrierten und ungetesten Features und TODOs für die zukunft

	\sectionlevelfour{Algorithmen}
	Verwaltung der Beschleunigungsstruktur ist der Löwenanteil, nicht die physiksimulation, die eher ein Dreizeiler 
	ist;
		\sectionlevelfive{Work Efficient Parallel Prefix Sum}
		\sectionlevelfive{Parallel Radix Sort und Stream Compaction}
		
		\sectionlevelfive{Ablauf}
			initialisierung, und beschreibung der einzelnen phasen...

		

\clearpage
