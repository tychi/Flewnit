Im Rahmen dieser Bachelorarbeit wurde der Frage nachgegangen, inwiefern eine sogenannte \"Unified Rendering-Engine\", welche verschiedene Simulationsdom{\"a}nen 
vereint, einen Mehrwert darstellen kann gegenüber dem klassischen Ansatz, z.B. eine Graphik- und eine Physik-Engine zu verwenden, die zun{\"a}chst mal keinen
Bezug zueinander haben; \\
Hierbei wurde besonderer Wert auf die Verwendung moderner GPU-Computing-APIs gelegt, namentlich auf OpenGL3/4 und OpenCL.\\
Ziel war vor allem das Potential und die Erweiterbarkeit des Frameworks, nicht die schnelle Realisierung eines grossen Feature-Sets.

\subsection{Motivation}

	Urspr{\"u}nglich als Arbeit zur Implementierung von Fluidsimulation geplant, wurde bald ein generalistischer, eher softwaretechnisch orientierter Ansatz verfolgt,
	der die Fluidsimulation jedoch als Endziel hatte;
	Der Wunsch nach einer \'Unified Rendering Engine\' erw{\"a}chst aus eigener Erfahrung der Kopplung von Physik- und Graphik-Engines, was einen gewissen Overhead mit sich 	bringt.... weitere Gr{\"u}nde: Liebe zu High-Performacne-Hardware, Wissensdurst in Bezug auf Rendering, Physik-Simulation, Hardware-Interna und Engine-Design,
	\'frei strampeln\' von sehr veralteten Technologien etc pp..  

\clearpage