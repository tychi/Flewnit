Bla bla

\subsection{Begrifflichkeiten}
	erklaren, was ich unter Unified Rendering verstehe, was Rendering dadruch fuer eine generalistische Bedeutung bekommt;

\subsection{Paradigmen}
	
	moeglichst symmetrischer Ansatz zwischen den Simulationsdomaenen, zantralistische Verwaltung von spezifischen Objekten,
	GPU-Code gernaierung per Template-Engine, moeglichst versuchen, langfristig so viele Aspekte wie moeglich miteinander kombinieren zu koennern 	    
	(Visualisierungstechniken, verschiedene Physik Partikelsimulation, Rigid bodiy-Simulation) 

\subsection{KlassenDiagramm}

\subsection{Dependencies}

	
	\subsubsection{OpenGL3/4}
    \subsubsection{OpenCL 1.0}
    \subsubsection{GLFW}
    	explizite GL3 core profile creation, einfaches fullscreen, multisampling, mouse grab, alles viel besser als GLUT :)
    \subsubsection{Grantlee}
       die string template engine die CL und GL code erzeugt
    \subsubsection{ogl math}
    	leichte, aber doch recht maechtige mathe-bibliothek
    	
    	
\subsection{Die Buffer-Abstraktion}   	
 	die bombe, die cpu, ogl und ocl vereint, inclusive ping ponging etc.. fundamentale Klassensammlung fuer den Unified-Aspekt
 
\subsection{Das WorldObject}
	Basis-Klasse fuer alles was unified simuliert wird: pure viuelle objekt, uniform grid, fluid, rigid body etc..
	erwaehnung des SubObjects;  
 
\subsection{Materials}  
	was stellt welches material in welcher Domain dar?
	
\subsection{Geometry}
	Abtract, Buffer bused, Vertex based etc.. ein paar konzepte (implementier/genutzt nut VertexBased)  
	
\subsection{Massively Parallel Program}
	Basisklasse von Shader und OpenCL Program
	\subsubsection{Shader}
		
	\subsubsection{OpenCLProgram}

weitere klassen/konzepte to go...	
	  	
  	

\clearpage