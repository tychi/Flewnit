
\subsection{Entwicklungsumgebung}

	Linux, CMake, git, Eclipse ; 
	gründe für auswahl:
		linux:  paketmanager für dependencies, unzufriedenheit mit microsofts stiefmütterlicher Behandlung von 64bit-		
			programmen, nicht zuletzt wunsch nach Vertiefung der kenntnisse der linux-welt
	
	cross platform build system, moderne versionsverwaltung, 	
 

\subsection{Klassendiagramm}

\subsection{Dependencies}

	
	\subsubsection{OpenGL3/4}
    \subsubsection{OpenCL 1.0}
    	noch keine offenen openCL 1.1 treiber, außerdem keine features davon benötigt;
    	c++-wrapper genutzt, auch von khronos-seiten beziehbar;
    	Gegenüberstellung zu CUDA, vor- und nachteile auflisten, insbesondere das problem ,dass esk ein 1D-texturen in OpenCL gibt, und man sich entscheiden muss zwischen generischem buffer und Textur, man also nicht hin und her-interpretieren kann wie in CUDA;
    \subsubsection{GLFW}
    	explizite GL3 core profile creation, einfaches fullscreen, multisampling, mouse grab, alles viel besser als GLUT :)
    \subsubsection{Grantlee}
       die string template engine die CL und GL code erzeugt
    \subsubsection{assimp}
    \subsubsection{ogl math}
    	leichte, aber doch recht maechtige mathe-bibliothek
    \subsubsection{TinyXML}
    	
    	
\subsection{Die Buffer-Abstraktion}   	
 	die bombe, die cpu, ogl und ocl vereint, inclusive ping ponging etc.. fundamentale Klassensammlung fuer den Unified-Aspekt
 
\subsection{Das WorldObject}
	Basis-Klasse fuer alles was unified simuliert wird: pure viuelle objekt, uniform grid, fluid, rigid body etc..
	
	\subsubsection{Das SubObject}
  
 
\subsection{Material}  
	was stellt welches material in welcher Domain dar?
	
\subsection{Geometry}
	Abtract, Buffer based, Vertex based etc.. ein paar konzepte (implementiert/genutzt nur VertexBased)  
	
\subsection{Massively Parallel Program}
	Basisklasse von Shader und OpenCL Program
	\subsubsection{Shader}
		
	\subsubsection{OpenCLProgram}

weitere klassen/konzepte to go...	
	  	
  	

\clearpage
