\label{sec:architecture:BufferAbstraction} 	

	Logischerweise müsste nun eigentlich 
	-- der Top-Down-Präsentation der einzelnen Pakete und Klassen entsprechend --
	die \lstinline|Geometry|-Klasse vorgestellt werden. Um letztere Klasse jedoch besser zu verstehen,
	ist die Kenntnis der Buffer-Abstraktion vorteilhaft. Die \lstinline|Geometry| wird in Abschnitt
	\ref{sec:geometry} beschrieben.
	
	Wie auf Seite \pageref{overview:bufferAbstraction} erwähnt, abstrahiert die \lstinline|BufferInterface|-Klasse
	sämtliche Buffer-bezogene Funktionalität, namentlich für Host-Buffer, OpenCL-Buffer, 
	verschiedene OpenGL-Buffertypen und letztendlich verschiedene Texturtypen (eine Teilmenge davon ist OpenCL-interop-
	kompatibel bzw. als reine OpenCL-Textur erschaffbar).
	Es ergibt sich ein wahrer Moloch, der für eine relativ einheitliche Handhabung zu abstrahieren ist: 
	
	\begin{itemize}
		\item verschiedene APIs (OpenGL C-Api "`vs."' OpenCL C++-API)
		\item verschiedene Verwendungen und Verfügbarkeiten schon allein von non-Textur-Buffern 
			(generisch, Vertex Index/Vertex Attribute Buffer, Uniform Buffer, Render Buffer),
			siehe Tabelle \ref{tab:bufferSupportInContexts}
		\item verschiedene OpenGL-Texturtypen mit nur bedingt kombinierbaren und nur teilweise zu OpenCL kompatiblen 
			erweiterten  Features, siehe Tabelle \ref{tab:textureTypes} 
			%(Tiefentextur ja/nein, MipMapping ja/nein, Textur-Array ja/nein, Multisample-Texture ja/nein, 
			%Rectangle Texture ja/nein, Cube Map ja/nein)
		\item verschiedene Datenformat/Channel-Layout-Deskriptoren %, siehe Listing \ref{listing:BufferElementInfo}
	\end{itemize}	
	
	Es hat schon sehr viel Zeit gekostet, sich aus den Spezifikations-Dokumenten 
	(\cite{openGL_4_1_Spec}, \cite{openCL_1_0_Spec}) die möglichen Permutationen, Zusammenhänge, Entsprechungen
	und unterstützten Operationen zu erarbeiten 
	(siehe Tabellen \ref{tab:bufferSupportInContexts} und \ref{tab:textureTypes}).
	
	Die Ergebnisse dieser Tabellen flossen in die Entscheidung ein, welche konkreten Buffer-
	und Textur-Klassen mit welchen  verfügbaren Features durch Ableitung vom
	\lstinline|BufferInterface| implementiert wurden. Die Konstruktor-Parameter sollten
	so spezifisch zugeschnitten sein, dass der Benutzer der Klassen möglichst wenig invalide
	Kombinationen angeben kann. Außerdem sollte er sich nicht um all die Makros
	wie \lstinline|GL_RGBA32UI| bzw \lstinline|CL_RGBA| und \lstinline|CL_UNSIGNED_INT32|
	kümmern müssen 
	\footnote{Was in OpenCL getrennt ist und somit "`nur"' 
		4 Channeltypen \emph{+} 3 Bitgrößen * (3 non-normalized + 2 normalized)= theoretisch 19 Makros ergibt,
		braucht bei OpenGL
			4 Channeltypen \emph{*} 3 Bitgrößen * (3 non-normalized + 2 normalized)= theoretisch 60 makros;
			Erstens stört die Asymmetrie, zweitens das "`Zusammenklatschen"' eigentlich unabhängiger Parameter in ein 
			Makro.}. 
	Es soll ein Buffer mit einem einzigen Konstruktor-Call vollständig definiert,
	falls erwünscht auf dem Host-Memory und -- sofern unterstützt --
	in den gewünschten API-Contexten allokiert bzw registiert werden.
	
	Hierfür mussten einige \lstinline|enum|s und Meta-Info-Klassen/Structures definiert werden:
	\begin{description}
		
		\item[enum Type] 
		Beschreibt String-, boolsche, Skalar-, Vektor- und Matrix-Datentypen mit verschiedener 
		Genauigkeit; findet in \emph{Flewnit} u.a. auch beim Config-Parsen Verwendung.
		
		%--------------------------------------------------------------------------------------------
		\item[enum ContextType und ContextTypeFlags] 
		Mithilfe dieser Datentypen lässt sich spezifizieren, in welchen
		Kontexten (Host, OpenGL, OpenCL) man einen Buffer erstellt/allokiert/registriert haben will.
		
		%--------------------------------------------------------------------------------------------
		\item[enum BufferSemantics] 
		\label{item:BufferSemantics}
		Eine wichtige Auflistung konkreter Verwendungszwecke des Buffers,
		siehe Listing \ref{listing:BufferSemantics} im Anhang \ref{append:Listings}.
		Dieser Aufzählungstyp ist eine große Erleichterung beim Hantieren mit OpenGL Vertex Buffer Objects (VBO's) 
		\footnote{abstrahiert von \lstinline|Flewnit::VertexBasedGeometry|, s. \ref{sec:geometry} }
		, da hier ein Index zur Assoziation von VBO-Vertex Attribute Buffern mit Input-Variablen im Vertex Shader
		angegeben werden muss. Dasselbe gilt für OpenGL Frame Buffer Objects (FBO's) 
		\footnote{abstrahiert von \lstinline|Flewnit::RenderTarget|} bei der Assoziation von
		an verschiedene Color-Attachments gebundenen Texturen mit den Output-Variablen eines Fragment Shaders.
		Wenn wir als Index die numerische Entsprechung der Buffer-Semantik verwenden, 
		ist nicht nur die Lesbarkeit des Codes erhöht, es ist auch automatisch sichergestellt, 
		dass Indizes keine Konflikte versursachen.
		
		%--------------------------------------------------------------------------------------------
		\item[enum GPU\_DataType] Listet nur die von der GPU / OpenGL nativ unterstützten Datentypen auf, ohne
		Präzisions- und Normalisierungs-Information:
		\begin{lstlisting}		
enum GPU_DataType
{
	GPU_DATA_TYPE_FLOAT,
	GPU_DATA_TYPE_INT,
	GPU_DATA_TYPE_UINT
};
		\end{lstlisting}
		
		%--------------------------------------------------------------------------------------------
		\item[enum GLBufferType]
		Listet die vom Buffer-Interface abstrahierten OpenGL-non-Textur-Buffertypen auf. 
		Aus Zeitgründen und Erwartung, dass sie außerhalb der \lstinline|RenderTarget|-Klasse nicht benötigt werden,
		werden z.B. Render Buffers nicht abstrahiert.
		\begin{lstlisting}		
enum GLBufferType
{
	NO_GL_BUFFER_TYPE,
	VERTEX_ATTRIBUTE_BUFFER_TYPE,
	VERTEX_INDEX_BUFFER_TYPE,
	UNIFORM_BUFFER_TYPE 
};
		\end{lstlisting}	
	
		%--------------------------------------------------------------------------------------------
		\item[struct BufferElementInfo] 
		Beschreibt -- sofern erwünscht bzw nötig -- das Channel-Layout,die Datentypen, die Bit-Genauigkeit
		und die etwaige Normalisierung eines Buffers mit Channels, 
		z.B. einer Textur oder einem Vertex Attribute Buffer.
		%siehe Listing \ref{listing:BufferElementInfo} im Anhang \ref{append:Listings}.
		Auf diese Weise spart man dem Benutzer die verschiedenen "`zusammengeklatschten"', asymmetrischen GL/CL-Makros;
		Die Structure validiert ihre Daten. Diese werden später für die internen CL/GL-API-Aufrufe in die
		entsprechenden CL/GL-Makros transformiert.
		%siehe Listing \ref{listing:BufferElementInfoToCLGL} im Anhang \ref{append:Listings}.
		
		%--------------------------------------------------------------------------------------------
		\item[struct GLImageFormat]
		OpenCL definiert folgende Structure:
		
		\begin{lstlisting}		
typedef struct _cl_image_format {
    cl_channel_order        image_channel_order;
    cl_channel_type         image_channel_data_type;
} cl_image_format;	
		\end{lstlisting}
		
		Um das interne Handling der CL/GL-Makros etwas symmetrischer zu machen (man verliert leicht den Überblick),
		ist eine ähnliche Structure definiert 
		\footnote{Obwohl beide APIs von der Khronos Group spezifiziert wurden und explizit 	
			Interoperabilität ermöglichen, bestehen viele kleine Asymmetrien zwischen den verschiedenen Makros mit
			eigentlich gleicher Semantik. Diese Umstände trugen zu der Entscheidung bei, 
			das \lstinline|BufferInterface| zu entwickeln, trotz des erheblichen Zeitaufwands.}
		
		
		\item[class BufferInfo]
		Dies ist die Klasse, die sowohl als "`Construction Info"' den Konstruktor-Code delegiert, als auch für 
		den Benutzer später Meta-Information bereitstellt.
		Es sei hier nur der Standard-Konstruktor angegeben:
		
		\begin{lstlisting}	
explicit BufferInfo(String name,
	ContextTypeFlags usageContexts,
	BufferSemantics bufferSemantics,
	Type elementType,
	cl_GLuint numElements,
	const BufferElementInfo& elementInfo,
	GLBufferType glBufferType = NO_GL_BUFFER_TYPE,
	ContextType mappedToCPUContext = NO_CONTEXT_TYPE);
		\end{lstlisting}
		
		Es sei gestanden, dass \lstinline|elementType| und \lstinline|elementInfo| redundant sind, außerdem diese 
		beiden Parameter eigentlich bei generischen Buffern, wo die Typ-Information zumindest für die GPU Computing APIs
		keine Rolle spielt, fehl am Platze sind. Diese Design Flaws sind mangelndem Überblick in der Anfangsphase der 
		Implementierung zu verdanken, auf welche verschiedenen Arten man Buffers in CL und GL nutzen kann. 
		Ein Refactoring mit spezieller zugeschnittenen Konstruktoren ist geplant.
	
	\item[class TextureInfo]
		So wie \lstinline|Texture| von \lstinline|BufferInterface| erbt, erbt auch 
		\lstinline|TextureInfo| von \lstinline|BufferInfo|. Sie stellt die gleiche Funktionalität für
		die (abstrakte) \lstinline|Texture|-Klasse dar wie \lstinline|BufferInfo| für \lstinline|BufferInterface|.
		Ein Unterschied besteht jedoch darin, dass \lstinline|TextureInfo| eigentlich nur dazu dient, 
		die kombinierte Information aus den Parametern der "`eingeschränkten"' Konstruktoren 
		der konkreten Textur-Klassen (\lstinline|Texture2D| etc.) und der inhärent klassenspezifischen
		Eigenschaften an die \lstinline|Texture|-Oberklasse zu übergeben. Damit hat \lstinline|TextureInfo|
		eher eine Textur-interne Funktion bei der Konstruktion. Das tut jedoch seiner Funktion als späterer
		Meta-Daten-Lieferant keinen Abbruch.
		Auch hier reicht der Standard-Konstruktor für eine Idee dieser Klasse:
		 
		\begin{lstlisting}	
explicit TextureInfo(
	const BufferInfo& buffi,
	cl_GLuint dimensionality,
	Vector3Dui dimensionExtends,
	GLenum textureTarget,
	bool isDepthTexture = false,
	bool isMipMapped = false,
	bool isRectangleTex = false,
	bool isCubeTex = false,
	GLint numMultiSamples = 1,
	GLint numArrayLayers = 1
	);
		\end{lstlisting}			
			
	\end{description}	

 	
 	\begin{table}[!h]
  		\begin{tabular}
  		{
  		 l  l | c | c | c |
  		}
																	\cline{3-5}
  									&								&	\multicolumn{3}{ c | }{Context} \\ 
  																	\cline{3-5}
									&								& 	Host 	& 	OpenGL 	& 	OpenCL	\\
    	\noalign{\hrule}								
    	\multicolumn{1}{|c|}{
    		generic Buffer
    	}							& 								
    		&	{\color{green}\checkmark} 	&	{\color{red}x}		& 	{\color{green}\checkmark}	\\ 
    	
    	\noalign{\hrule}								
    	\multicolumn{1}{|c|}{
    		\multirow{4}{*}{OpenGL Buffers}
    	}							& Vertex Attribute Buffer		
    		&	{\color{orange}o} 	&	{\color{green}\checkmark}		& 	{\color{orange}o}	\\  
    								\cline{3-5}
    	\multicolumn{1}{|c|}{}		& Vertex Index Buffer			
    		&	{\color{orange}o} 	&	{\color{green}\checkmark}		& 	{\color{orange}o}	\\  
    								\cline{3-5}
    	\multicolumn{1}{|c|}{}		& Uniform Buffer
    		&	{\color{orange}o} 	&	{\color{green}\checkmark}		& 	{\color{orange}o}	\\ 
    								\cline{3-5} 
    	\multicolumn{1}{|c|}{}		& Render Buffer					
    		&	{\color{red}x} 	&	{\color{green}\checkmark}		& 	{\color{green}\checkmark}	\\ 
    
   		\noalign{\hrule}								
   		\multicolumn{1}{|c|}{
    		\multirow{4}{*}{Textures} 
   		}							& 1D Texture					
   			&	{\color{orange}o} 	&	{\color{green}\checkmark}		& 	{\color{red}x}	\\ 
    								\cline{3-5}
		\multicolumn{1}{|c|}{}		& 2D Texture				
			&	{\color{orange}o} 	&	{\color{green}\checkmark}		& 	{\color{green}\checkmark}	\\ 
									\cline{3-5}
		\multicolumn{1}{|c|}{}		& 3D Texture		
			&	{\color{orange}o} 	&	{\color{green}\checkmark}		& 	{\color{green}\checkmark}	\\ 
									\cline{3-5}
		\multicolumn{1}{|c|}{}		& Special Texture				
			&	{\color{orange}?} 	&	{\color{green}\checkmark}		& 	{\color{orange}?}	\\ 


    	\noalign{\hrule}
     
     	
  		\end{tabular}	
  	
  		\caption{		
  			Verschiedene Buffertypen und ihre Verfügbarkeit in verschiedenen Kontexten \\	
  			Legende: \\
			{\color{green}\checkmark}	$\rightarrow$ nativ unterstützt;
			{\color{orange}o}	$\rightarrow$ kompatibel;
			{\color{red}x}	$\rightarrow$ nicht unterstützt;	\\
			{\color{orange}?}	$\rightarrow$ Unterstützung abhängig von weiteren Parametern, 
								s. Tabelle \ref{tab:textureTypes};	
		}
		\label{tab:bufferSupportInContexts}
  	\end{table}
  	
  	
  	\begin{table}[!h]
  		\begin{tabular}
  		{
  			|l|c|c|c|c|c|
  		}
		\noalign{\hrule}						
  						&	CL interop	&	MipMap	& Depth	&	Array	&	Rectangle	\\
  		\noalign{\hrule}						
  		Texture1D		& {\color{red}x} & {\color{green}\checkmark} & {\color{red}x}
  						& {\color{green}\checkmark} & {\color{red}x} \\
		\noalign{\hrule}						
  		Texture2D		& {\color{green}\checkmark} & {\color{green}\checkmark} & {\color{green}\checkmark}  
  						& {\color{green}\checkmark} & {\color{green}\checkmark} \\
  		\noalign{\hrule}						
  		Texture2DCube 	& {\color{yellow}o} & {\color{green}\checkmark}  & {\color{green}\checkmark}
  						& {\color{orange}o} & {\color{red}x} \\
  		\noalign{\hrule}						
  		Texture2DMultiSample& {\color{red}x} & {\color{red}x}  & {\color{red}x} 
  						& {\color{green}\checkmark} & {\color{red}x} \\	
  		\noalign{\hrule}						
  		Texture3D		& {\color{green}\checkmark} & {\color{green}\checkmark}  & {\color{red}x} 
  						& {\color{red}x} & {\color{red}x} \\
  		\noalign{\hrule}						
  		\end{tabular}
  		
  		\caption{Verschiedene Texturtypen und ihre Kompatibiltät zu bestimmten Features \\	
  			Legende: \\
			{\color{green}\checkmark}	$\rightarrow$ unterstützt;
			{\color{yellow}o}	$\rightarrow$ In OpenCL unterstützt, aber nicht vom Framework;
			{\color{orange}o}	$\rightarrow$ nur in OpenGL 4 unterstützt,
								daher wegen Kompat. zu GL 3 nicht vom Framework unterstützt;
			{\color{red}x}	$\rightarrow$ nicht unterstützt;
		}
  		\label{tab:textureTypes}
	\end{table}
	
	Es sei bemerkt, dass die Tiefen-Texturen sich abgesehen von den automatisch bestimmten
	entsprechenden OpenGL-Makros in ihrer Implementation nur geringfügig von ihren "`Farbtextur"'-
	Oberklassen unterscheiden, nämlich in der Implementation der 
	\lstinline|virtual void setupDefaultSamplerParameters()=0;|-Methode, die in \lstinline|Texture| definiert ist.
	Hier wird z.B. der Compare-Mode und die Compare-Function für korrekten Lookup solcher Texturen als Shadow Map
	über einen	Shadow-Sampler in GLSL gesetzt.
	
	Man darf sich fragen, warum jede noch so sonderbar anmutende Spezial-Textur unterstützt werden soll. Dies
	hat den Grund, dass mittelfristig etliche Rendering Features implementiert werden soll, welche die meisten dieser 	
	Spezialtexturen benötigen, und die verbleibenden paar ungenutzten Texturen noch zur Vollständigkeit mit zu 
	implementieren, verursacht dank Vererbung nur einen minimalen Mehraufwand:
	\begin{enumerate}
		\item Layered Rendering in \lstinline|Texture2DDepthCube| zur Generierung von Point Light Shadow Maps
		in nur einem Rendering Pass
		\item Layered Rendering in \lstinline|Texture2DCube| zur Generierung von dynamischen Environment Maps
		\item  Layered Rendering in \lstinline|Texture2DDepthArray| zur Generierung von vielen Spot Light Shadow Maps
		in nur einem Rendering Pass
		\item Deferred Shading mit Multisampling: Benötigt \lstinline|Texture2DMultiSample|-G-Buffers
	\end{enumerate}
	Zu weiten Teilen sind der Shadercode und die nötigen Framework-Klassen schon für diese Features implementiert,
	es muss "`nur"' noch einiges ergänzt und aufgeräumt, alles zusammengefügt und debuggt werden. 
	Da dieses "`Nur"' wirklich äußerst ironisch gemeint ist, wurde gar nicht erst versucht, 
	die Implementierung dieser Features zu beenden. Es muss vorerst reichen,
	dass sie konzeptionell in der Framework-Struktur angelegt sind.\\
	
	
	%------------------------------------------------------------------------------------------
	
	Alle Operationen auf den Buffern/Texturen durch den Benutzer werden dann über das \lstinline|BufferInterface|
	getätigt, siehe das exemplarische Listing \ref{listing:bufferOperationInterface}. 
	Fast sämtlicher Kontrollfluss (Synchronisation,Validierung, Auswahl des richtigen Kontextes etc.)
	befindet sich in der Implementation	der (wohlgemerkt großteils \emph{nicht} virtuellen) 
	Methoden von \lstinline|BufferInterface|. 
	Nur die finale Bürde der verschiedenen API-Calls als Ergebnis des vorangegangenen Kontrollflusses
	wird von \lstinline|protected| virtuellen Funktionen der abgeleiteten Klassen übernommen, wie im folgenden 	
	Unterabschnitt beschrieben.
	
\begin{lstlisting}[caption={Operationen auf dem BufferInterface, Ausschnitt},label=listing:bufferOperationInterface]		
	//The binding-machanism is only relevant for OpenGL	
	void bind()throw(BufferException);
	void setData(const void* data, ContextTypeFlags where)throw(BufferException);
	//if both CL and GL are enabled, then the buffer is shared and the implementation
	//will decide which API will be used for the read/write;
	void copyFromHostToGPU()throw(BufferException);
	void readBack()throw(BufferException);
\end{lstlisting}
	
	\subsubsection{Die API-bezogenen internen Buffer-Operationen}
	Zur Vermeidung von Boilerplate-Code sind die Signaturen für die GPU Computing API-bezogenen 
	Buffer-Operationen ausgelagert in eine Datei namens "`BufferVirtualSignatures.h"'.
	Um sowohl in der abstrakten Basisklasse als auch in den abgeleiteten Klassen included werden zu können,
	lässt sich die Abstraktheit der Definitionen über das Makro \lstinline|FLEWNIT_PURE_VIRTUAL| steuern:
		
	\begin{lstlisting}[caption={API- und Buffertyp-abhängige Operationen auf Buffern -- Definitionen},
					   label=listing:bufferOpDefs]	
#ifdef FLEWNIT_PURE_VIRTUAL
#	define PURENESS_TAG =0
#else
#	define PURENESS_TAG
#endif

	virtual void generateGL() PURENESS_TAG;
	virtual void generateCL() PURENESS_TAG;
	virtual void generateCLGL() PURENESS_TAG;

	//there is no pendant to bind() in OpenCL, as the API needs no buffer binding mechanism;
	//instead, buffers are passed as arguments to kernels
	virtual void bindGL() PURENESS_TAG;
	//there is no pendant to alloc() in OpenCL, as Buffer generation and allocation are coupled within
	//the same API call
	virtual void allocGL()PURENESS_TAG;

	virtual void writeGL(const void* data) PURENESS_TAG;
	virtual void writeCL(const void* data) PURENESS_TAG;
	virtual void readGL(void* data)	PURENESS_TAG;
	virtual void readCL(void* data)	PURENESS_TAG;
	virtual void copyGLFrom(GraphicsBufferHandle bufferToCopyContentsFrom) PURENESS_TAG;
	virtual void copyCLFrom(ComputeBufferHandle & bufferToCopyContentsFrom) PURENESS_TAG;
	//because the C++ Wrapper for OpenCL is used, deletion takes place automatically when th cl::Buffer is destroyed
	virtual void freeGL() PURENESS_TAG;
	
//	//not implemented yet as not needed at the moment: mapping routines:
//	virtual void* mapGLToHost() PURENESS_TAG;
//	virtual void* mapCLToHost() PURENESS_TAG;
//	virtual void* unmapGL() PURENESS_TAG;
//	virtual void* unmapCL() PURENESS_TAG;

#undef PURENESS_TAG
	\end{lstlisting}
	
	So included \lstinline|BufferInterface| diese Datei innerhalb ihrer Klassendefinition über
	\begin{lstlisting}
#	define FLEWNIT_PURE_VIRTUAL
#	include "BufferVirtualSignatures.h"
#	undef FLEWNIT_PURE_VIRTUAL
	\end{lstlisting}
	Die anderen Klassen definieren das Makro nicht vor dem Includen.\\
	Bei den Textur-Klassen werden die meisten Signaturen aus "`BufferVirtualSignatures.h"'
	schon von der Oberklasse \lstinline|Texture| implementiert, da nicht jede API-Routine Textur-spezifisch ist.\\

	
	\subsubsection{PingPongBuffer}
	Auch die für viele Algorithmen aufgrund ihrer inhärenten Struktur oder aufgrund der Beschränkungen bei der 	
	Synchronisation parallelen Codes nötige "`Ping-Pong"'-Funktionalität \footnote{also die Verwendung von zwei Buffern, 
	einem zum Lesen, einem zum Schreiben, und die Rollen werden nach jedem Schritt getauscht} ist durch das
	\lstinline|BufferInterface| abstrahiert: Der \lstinline|PingPongbuffer| verwaltet intern zwei andere
	\lstinline|BufferInterface|-Objekte und leitet die Methoden-Aufrufe, die durch das \lstinline|BufferInterface| 
	definiert sind,	an den aktuell aktiven Buffer weiter. Der Rollentausch geschieht dann über 
	\lstinline|PingPongBuffer::toggleBuffers()|.
	Bei komplexeren Algorithmen, bei denen viele Toggles passieren, kann man schnell den Überblick verlieren, welcher
	Buffer jetzt gerade aktiv ist; Besonders kritisch ist dieser Umstand dort, wo ein OpenGL Buffer-Handle
	gebunden ist, oder ein OpenCL-Buffer-Handle als Argument für einen OpenCL-Kernel gesetzt ist.
	Diese Bindings lassen sich nicht von einem \lstinline|toggleBuffers()| beeindrucken, 
	da die \lstinline|BufferInterface|-Klasse selbst nicht weiß, in in welche Hierarchie an verschiedensten Binding-Points
	seine verwalteten Handles gerade integriert sind oder nicht.
	Um dem Programmierer die Bürde zu nehmen, neben den Toggles auch noch ständig die CL/GL-Bindings zu ändern,
	wurden Mechanismen implementiert, die diese Operationen automatisch tätigen, nämlich bisher in 
	\lstinline|VertexBasedGeometry| um Attribute und Index Buffer Bindings konsitent zu halten 
	(s. Abschnitt \ref{sec:VertexBasedGeometry}) und in
	\lstinline|CLKernelArgument| (s. Abschnitt \ref{sec:CLKernelArguments}), um die Kernel-Arguments nicht "`von Hand"'
	immer neu setzen zu müssen.
	