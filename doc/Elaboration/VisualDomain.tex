
\label{sec:visualDomain}
	
Ein paar worte ueber die shading features, wie sie maskiert werden etc..

\subsubsection{Der LightingSimulator}
	Nochmal drauf hinweisen, dass Rendering etwas generisches in diesem Framework ist, und wir leiber von Lichtsimulation sprechen sollten, auch wenn es monetan nicht photrealistisch ist ;)

\subsubsection{Die Lighting Simulation Pipeline Stages}
	shadowmap gen stage, direct lighting stage, was noch in planung is etc..	

\subsubsection{ShaderManager}
	generiert mit grantlee, assigned an materials und verwaltet Shader , abhaenging von der aktuellen lighting stage, den registierten Materials,
	der Erzeugten kontext, den vom user aktivierten rendering features etc pp

\subsubsection{genutzte moderne OpenGL- Features}	

	\sectionlevelfour{Uniform Buffers}
	auch von BufferInterface abstrahiert, vorteile auflisten, aber auch stolperfallen: alignment etc)	

	\sectionlevelfour{Hardware Tesselation}
	basics des hardware features erwaehnen fuer den geneigten leser, raptor-modell erwaehnen und seinen 		
	Aufbereitungsprozess, LOD, displacement mapping erlaeutern	
	
	\sectionlevelfour{Instancing}
	InstanManager, InstangedGeometry vorstellen, konzept, wie es verwaltet wird, erklaeren
	
	\todo{continue brainstorming}

	
	  

\clearpage
