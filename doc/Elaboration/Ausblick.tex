



	Zunächst möchte ich ergänzende Information zu dem Ansatz und der Motivation der
	"`static triangle collision meshes"' liefern, die in Abschnitt \ref{sec:statusImplementation} 
	angedeutet wurden:\\
  	Mich hat es schon während der Recherche-Phase gestört, dass ich keine Information gefunden habe, ob und wenn ja, 
  	wie primitiv-basierte Rigid Body-Simulation -- also eine Simulation, wo die Rigid Bodies 
  	aus Dreiecken oder komplexeren Primitiven aufgebaut sind und nicht aus Partikeln --
  	auf der GPU effizient möglich ist. Ich habe nur Papers
  	zu CPU/GPU-hybriden Ansätzen oder partikel-repräsentierten Rigid Bodies gefunden (letzteren Ansatz
  	habe ich selber in meinem ungetesteten Code verwendet). 
  	Ich wollte zumindest für statische Dreiecks-Meshes testen, ob diese,
  	da sich ihre räumliche Position und damit auch die Postion im Speicher nicht mehr ändert,
  	effizient auf die GPU zu mappen sind. Hierfür habe ich einen Algorithmus entwickelt, der beschreibt,
  	wie man das Mesh überhaupt erst einmal Artefakt-frei aufbereitet für die Integration in ein Uniform Grid 
  	(über Voxelisierung von	Primitive- ID's, wobei der Umstand zu berücksichtigen ist, 
  	dass ein Vertex zu beliebig vielen Dreiecken gehören kann).
  	Die nächste Herausforderung, wenn die Primitive-ID's wie Particle-ID's über das Uniform Grid verfügbar sind,
  	ist die Datenstruktur für die Dreiecke und eine effiziente Kollisions-Berechnung zwischen Dreick und Partikel.
  	Hierüber kann womöglich die Ray Tracing-Literatur mit ihren vielen Ansätzen zu effizienten
  	Dreiecks-Repräsentationen und Strahl-Primitiv-Schnitttests Aufschluss geben; Beim Ray Tracing ist der effiziente 
  	Strahl-Primitiv-Test noch fundamentaler.\\
  	Der Mangel an Literatur ist kein gutes Omen für die Effizienz des Ansatzes. Dennoch finde ich es
  	interessant, ihn weiter zu verfolgen, bis Klarheit über die effiziente Realisierbarkeit herrscht.\\
  	
  	
  	
  	In Abschnitt \ref{sec:statusImplementation} wurden die gröbsten Unvollständigkeiten der Implementierung
	bereits aufgezeigt. Dieses Kapitel fasst die fehlenden Features zu einer priorisierten
	"`to do"'-Liste zusammen:\\
	\begin{enumerate}
		\item Grund für den auf Seite \pageref{enum:oclSyncBug} beschriebenen Bug herausfinden und ihn fixen,
		sofern es kein Treiber-Bug ist
		\item Performance und Stabilität der Optimierung testen, welche Dichte-Berechnungen im selben
		Kernel ausführt wie die Kraft-Berechnungen (die Stabilität kann insofern leiden, dass bei diesem Ansatz
		immer die Dichte-Werte vom vorherigen Simulationsschritt verwendet werden müssen, welche somit veraltet sind)
		\item Das Access Pattern für Nachbar-Partikel von Goswami (s. S. \pageref{enum:goswamiAccessPattern})
			 auf linearen Buffern mit Fermi-Devices auf Performance prüfen, bei negativem Ergebnis Pattern verwerfen
			 oder Tricks mit 2D-Texturen erwägen, außerdem versuchen, mehr über die L1-Cache-Nutzung in Nvidia-Hardware
			 herauszufinden
		\item Die partikelbasierte Rigid-Body-Simulation lauffähig machen
		\item Ein anderes Integrations-Schema als das sehr Bandbreiten-lastige und speicherhungrige 
			\emph{Velocity Verlet}-Verfahren testen.
			%Da das \emph{Leap Frog}-Verfahren seine Vorteile verliert,
			%sobald man Geschwindigkeits-Werte zur Kraftberechnung benötigt (wie etwa bei Viskositäts-Berechnungen),
			%muss ein geeignetes Schema, welches einen guten Kompromiss zwischen Genauigkeit und Performance
			%erst noch gefunden werden.
		\item Das Fluid angemessen visualisieren nach \cite{Green2009FluidRenderingCurvatureFlow}, aus Erkenntnissen
		hieraus womöglich die \lstinline|Shader|-Klasse und den Shader-Template-Code des
		"`GenericLightingUberShader"' refactorn
		\item Die Funktionalität, mehrere Fluide zu simulieren und visualisieren, lauffähig machen
		\item OpenCL-Kernels profilen, Flaschenhälse finden, optimieren
		\item Den \lstinline|SceneLoader| fertig implementieren
		\item Die Features zu Ende implementieren, für welche schon viel Code im System liegt: 
		Deferred und Layered Rendering
		\item Funktionalität zur Voxelisierung von Geometrie bereitstellen, so dass beliebige "`water tight"' Meshes
		an der Rigid Body-Simulation teilnehmen können
		\item Die "`static triangle collision meshes"' implementieren
		\item Eine GUI implementieren
		\item In Hinblick auf Vision der Paddel-Simulation: \emph{Wii Remote}-Anbindung inklusive \emph{Wii MotionPlus}-
		Funktionalität\footnote{http://en.wikipedia.org/wiki/Wii\_Motion\_Plus} realisieren
		\item weitere Features implementeren: Ambient Occlusion, Depth of Field etc.
		\item "`akustische Simulation"': testen, ob die generische Framework-Struktur sich auch auf diese Domäne
			übertragen lässt, und ob dies einen Mehrwert darstellt
	\end{enumerate}

	Es gibt genügend für die Zukunft zu tun. Es wird definitiv ein Langzeit-Projekt.
	Hoffentlich bleibt ein wenig Luft für die Weiterentwicklung.


\clearpage
